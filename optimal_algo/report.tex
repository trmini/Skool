\documentclass[fullpage,my,12pt]{article}
% DT -- if you are working on UNIX, change this to
\input{psfig}
%\input vpsfig.sty

\usepackage{my}
\usepackage{fullpage}
\newcommand{\same}{{\cal Q}}
\newcommand{\trivial}{${\cal A}_{trivial}$}
\newcommand{\qpos}{{\rm\mbox{``$+$''}}}

\begin{document}

\title{An $O(n \log{n})$ Algorithm for Constructung a
Consistent Convex $k$-gons. (Extended Abstract)}
\author{Stephen Kwek, Dien Trang Luu\thanks{This research is
supported by UTSA Faculty Research Award 14-7518-01.} \\
Division of Computer Science \\
University of Texas at San Antonio \\
San Antonio, TX 78249 \\
Email: kwek@cs.utsa.edu, dluu@cs.utsa.edu } \date{\today}
\maketitle

\begin{abstract}
... A short summary of results. DT - you will fill this in later
when I have done with the intro. and previous work section.


\end{abstract}
\thispagestyle{empty}
 
\noindent
{keywords: PAC Learning, Computational Geometry, Consistent
Algorithm}

\newpage

\setcounter{page}{1}
\section{Introduction}
I will fill it in later.

\section{Previous Results}
I will fill it in later.

\section{Preliminary Lemmas}
Through out our discussion, we will use the following notation:

$S$ -- the set of labeled sample points.

$n$ -- the size of $S$.

$S^+$ -- the subset of $S$ that are labeled positive.

$S^-$ -- the subset of $S$ that are labeled negative.

$conv(A)$ -- the convex hull of the set of points $A$.

$k-gon$ -- a convex polygon with $k$ sides (edges).

$line(e)$ -- the (extended) line containing the edge $e$.

The output of our algorithm is a $k$-gon such that each edge may
be incident to a negative point $p^-$ and a positive point $p^+$.
By performing a very small perturbation of the edges, we can
obtain a polygon that have the negative point $p^-$ in the
exterior while the positive point $p^+$ in the interior, without
changing an interior point to an exterior point or vice versa.

\begin{observation}~\label{obv:enclose}
A consistent polygon must contain $conv(S^+)$.
\end{observation}
\pf
A consistent polygon $P$ contains $S^+$. Therefore $P$ must contain the
convex hull of $S^+$, $conv(S^+)$.
\qed

A {\em supporting line} of a polygon $P$ is a line that contains
exactly a vertex or an edge of $P$.

WLOG, we assume the points in $S^-$ are labeled as $p_1, \ldots,
p_i, \ldots, p_l$ in ascending order of their polar angles around
some arbitrary point in $conv(S^+)$.

%\begin{figure}
%\mbox{\psfig{figure=supporting.eps,width=6.00in}}
%\caption{There are two supporting lines
%of $conv(S^+)$ that pass through $p_i$,
%$p_i \in S^-$.}\label{supporting}

\begin{figure}
\begin{center}
\input{./lemma4_fig.eps}
\end{center}
\caption{There are two supporting lines
of $conv(S^+)$ that pass through $p_i$,
$p_i \in S^-$.}
\label{supporting}
\end{figure}

Let $p_i$ be an arbitrary point in $S^-$. Then there
are two supporting lines {\em left}, $l_i^l$, and {\em right}, $l_i^r$,
that pass through the point $p_i$ (See Figure~\ref{fig:supporting})
and lean on $conv(S^+)$. We also define,
\begin{eqnarray*}
c_l &=& l_i^l \cap conv(S^+)\\
c_r &=& l_i^r \cap conv(S^+)
\end{eqnarray*}
   
\begin{lemma}~\label{lem:supporting}
The region in the opposite of the quadrant holding $conv(S^+)$
does not contain any consistent polygon vertex.
\end{lemma}

\pf
Suppose one of the consistent polygon vertces, $c$ is in the
forbidden region. Point $c$ will form a triangle with $c_l$ and $c_r$.
Since $c$ is a vertex of $P$ and $c_l$, $c_r$ $\in$ $S^+$, $c$,
$c_l$, and $c_r$ are positive. Hence, all points in $\triangle c c_l c_r$
are positive which implies $p_i$ being inside $\triangle c c_l c_r$
has to be positve. This contradict $p_i \in S^-$.
\qed

By Lemma~\ref{lem:supporting}, each point $p_i \in S^-$ gives
raise to a {\em forbidden region} in which no vertex of a
consistent polygon can fall. Clearly, the union of all these
forbidden regions forms an exterior of a star-shaped polygon
$P^s = p_1 p'_1 p_2 p'_2 ... p_l p'_l$. Here $p'_i$ is the
intersection of a supporting line of $p_i$ with another
supporting line of $p_{i-1}$ [I need to define this better, will
do it later]. Clearly, we have the following observation. We call
the set of edges $\{\overline{p_ip'_i}\}$ the {\em left edges}
(of $p_i$) while the other edges $\overline{p_ip'_{i+1}}$ the
{\em right edges} (of $p_i$).


\begin{observation}~\label{lem:between}
All the vertices must lie in the region $P^s - conv(S^+)$. That
is, they must lie in the region between $conv(S^+)$ and $P^s$.
\end{observation}

\begin{lemma}~\label{lem:onePoint}
There is a consistent polygon such that each of its edges
contains a vertex of $conv(S^+)$.

\end{lemma}
\pf
We can draw a supporting line from $c_i$ to one convex of $conv(S^+)$.
This line intersects the polygon edge started from $c_{i+1}$ at $c_{i+1}^'$.
Point $c_i$, $c_{i+1}$, and $c_{i+1}^'$ form a triangle, $\triangle c_i c_{i+1} c_{i+1}^'$,
which does not contain any positive points. Therefore, we can cut off this triangle
to from a consistent polygon.
\qed

\begin{lemma}~\label{lem:leftEdge}
There is a consistent polygon $P$ such that each vertex, except
possibly one, lies on a left edge of $P^s$.
\end{lemma}
\pf
Extending the line $e_{i-1}$ to intersect the supporting line $l_i^r$, we have
a point $c_i^'$. The line $l_i^r$ intersects $e_{i+1}$ at $c_{i+1}^'$. The
line $c_i^' c_{i+1}^'$ will be the edge of a consistent polygon. Because
$c_i^'$ and $c_{i+1}^'$ are still on their original lines, i.e $e_{i-1}$ and $e_{i+1}$,
the line and other edges of $P$ can from a consistent polygon.
\qed

\section{An Efficient Algorithm for Finding a Consistent Polygon}

\begin{lemma}~\label{lem:compStar}
$P^s$ can be computed in $O(n \log n)$ time.
\end{lemma}
\pf
DT(5) -- fill this up.

\qed

The proof of Lemma~\ref{lem:leftEdge} also implies that the
possible candidates of the vertices are intersection of an
extended line containing a left edge and another left edge. Let
$E_i$ denote the left edge $\overline{p_ip'_i}$, and $C(E_i)$ be
the set of possible vertex candidates on $E_i$. In other words,
$C(E_i)$ is the set $\{p: p = E_i \cap line(E_j), i\not=j. \}$

\begin{lemma}~\label{lem:compCandidate}
All the $C(E_i)$'s can be computed in $O(n \log n)$ time.
\end{lemma}
\pf
DT(6) -- Perform the kind of binary search that we were
discussing. In fact, it may be possible to do it in $O(n)$ time.
However, we are only require to come up with an $O(n \log n)$
time result. You will have to use Lemma~\ref{lem:leftEdge}.

\qed

\begin{theorem}~\label{thm:consistent}
The consistent polygon problem can be solved in $O(n \log n)$
time.
\end{theorem}
\pf
DT(7) -- filled it up. Complete the proofs of the earlier lemmas
first and then meet me to discuss about this.


Skeleton: \\
1. Construct $conv(S^+)$ \\
2. Construct $P^s$ \\
3. For each leftedge $E_i$, find the set of left edges $E_j$ such
that the extended line intersects it.
4. Compute how "far" you can reach as discussed before.

\qed

\end{document}
