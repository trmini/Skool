\documentstyle{article}

\begin{document}

\noindent
{\bf Problem 4}\\

We have $s_k(\Delta)$ and $m_k(\Delta)$ as follows,\\
\begin{eqnarray*}
s_k(\Delta) &=& \frac{1}{k}\sum_{t=\alpha+1}^{\alpha+k}\omega(t,\Delta) \\
m_k(\Delta) &=& \frac{1}{k}\sum_{t=\alpha}^{\alpha+k-1}\omega(t,\Delta)
\end{eqnarray*}
{\bf a.} At first,\\
\begin{eqnarray*}
  \omega(t,\Delta) &=& \omega(t+1,\Delta+1) - \delta(t,\Delta) \\
  \Rightarrow \omega(t+1,\Delta+1) &=& \omega(t,\Delta) + \delta(t,\Delta)
\end{eqnarray*}
Summing all $\omega(t, \Delta+1)$ from $t=\alpha$ to $t=\alpha+k$ we have, \\
\begin{eqnarray*}
  \frac{1}{k}\sum_{t=\alpha}^{\alpha+k}\omega(t+1,\Delta+1) = \frac{1}{k}
  \sum_{t=\alpha}^{\alpha+k-1}(\omega(t,\Delta) + \delta(t,\Delta) \Rightarrow \\
  s_k(\Delta+1) = m_k(\Delta) + \frac{1}{k}\sum_{t=\alpha+1}^{\alpha+k}
  \omega(t,\Delta) + \frac{\omega(\alpha, Delta)}{k} - \frac{\omega(\alpha+k,\Delta)}{k} \Rightarrow\\
  s_k(\Delta+1) = s_k(\Delta) + m_k(\Delta) - \frac{\omega(\alpha+k,\Delta)
  -\omega(\alpha, \Delta)}{k}
\end{eqnarray*}
{\bf b.} From the proof of {\bf a.} we have,\\
\begin{eqnarray*}
  s_k(\Delta+1) &=& s_k(\Delta) + m_k(\Delta) - \frac{\omega(\alpha+k,\Delta)
  -\omega(\alpha, \Delta)}{k} \\
  s_k(\Delta-1) &=& s_k(\Delta) - m_k(\Delta-1) + \frac{\omega(\alpha+k,\Delta
  -1) -\omega(\alpha, \Delta-1)}{k}
\end{eqnarray*}
Adding the above two terms we have,\\
\begin{eqnarray*}
  s_k(\Delta+1) + s_k(\Delta-1) &=& 2s_k(\Delta) + (m_k(\Delta) - m_k(\Delta-1))\\
  & & - \frac{\omega(\alpha+k,\Delta) -\omega(\alpha, \Delta)}{k}
  + \frac{\omega(\alpha+k,\Delta -1) -\omega(\alpha, \Delta-1)}{k} (\star)
\end{eqnarray*}
Let's consider the term $m_k(\Delta) - m_k(\Delta-1)$. We have
\begin{eqnarray*}
m_k(\Delta) - m_k(\Delta-1) &=& \frac{1}{k}\sum_{t=\alpha}^{\alpha+k-1}\delta(t,\Delta)
-\frac{1}{k}\sum_{t=\alpha}^{\alpha+k-1}\delta(t,\Delta-1) \\
\Leftrightarrow m_k(\Delta) - m_k(\Delta-1) &=& \frac{1}{k}\sum_{t=\alpha}^{\alpha+k-1}
(\delta(t,\Delta) - \delta(t,\Delta)) (\star \star)
\end{eqnarray*}
In case $\delta(t,\Delta-1) = 0$, the memory referece $r_{t+1} \in \omega(t,\Delta-1)$
which implies that $r_{t+1} \in \omega(t,\Delta)$, or $\delta(t,\Delta) = 0$ too.\\
The above implies that $\delta(t,\Delta-1)$ is greater than or equal to $\delta(t,\Delta)$,
or in another word $\delta(t,\Delta) \leq \delta(t,\Delta-1) \Rightarrow (\star \star) \leq 0$.\\
Apply $(\star \star) \leq 0$ back in $(\star)$ we have,\\
\begin{eqnarray*}
s_k(\Delta+1) + s_k(\Delta-1) &\leq& 2s_k(\Delta) - \frac{\omega(\alpha+k,\Delta)-\omega(\alpha,\Delta)}{k}
     + \frac{\omega(\alpha+k,\Delta -1)-\omega(\alpha,\Delta-1)}{k} \\
     \Leftrightarrow s_k(\Delta+1) + s_k(\Delta-1) &\leq& 2s_k(\Delta) +\frac{1}{k}[\omega(\alpha+k,\Delta-1)
     -\omega(\alpha, \Delta-1)-\omega(\alpha+k,\Delta)+\omega(\alpha,\Delta)]
\end{eqnarray*}
\end{document}
